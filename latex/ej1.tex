\section{Ejercicio 1: Laberinto}
    % 1. Describir detalladamente el problema a resolver dando ejemplos del mismo y sus soluciones.
    \subsection{Descripción del problema}
		    Indiana Jones continua en su expedición en la fortaleza de alguna civilización antigua. Mientras llenaban las mochilas con los tesoros que más les convenian encontraron un mapa peculiar. El mapa se parece mucho a un laberinto salvo por el hecho de que no estan conectados todos los puntos. En el mapa hay un punto que parece indicar el lugar en donde se encuentran juntando tesoros, y hay un lugar que esta indicado con una cruz. Intrigados por saber lo que se encuentra en ese lugar, se proponen como objetivo ir hacia ahi. Como nuestro equipo vino equipado con pico y pala, pueden derribar algunas paredes. Por otro lado el equipo ya se encuentra cansado y no desea recorrer mucha distancia ni esforzarse rompiendo paredes. Entonces, nos pidieron ayuda con lo siguiente: Teniendo un mapa indicando un punto de origen (identificado con una $o$) y un punto de destino (identificado con una $x$), quieren caminar lo menos posible desde el origen al destino rompiendo a lo sumo una cierta cantidad de paredes.

        La resolución del problema consiste en encontrar el camino más corto que derribe a los sumo $P$ paredes, las cuales se reconocen de un mapa dibujado con $.$ y $\#$ que indican los lugares por donde se puede pasar y las paredes respectivamente. Cada paso cuenta como 1 avance desde el origen y cuando una pared es derribada, se toma en cuenta como un lugar por el que se puede pasar, por lo que suma 1 también como parte del camino. El mapa tiene tamaño $F$ filas y $C$ columnas, los cuales se pasan por parametro al programa junto con la cantidad de paredes y el mapa. En caso de que este camino no exista, entonces la salida será solamente $-1$

        Por ejemplo, si el programa recibe lo siguiente como entrada: \newline
        \texttt{5} \texttt{9} \texttt{2} \newline
        \texttt{\# \# \# \# \# \# \# \# \#} \newline
        \texttt{\# o \# . \# . \# x \#} \newline
        \texttt{\# . \# . \# . \# . \#} \newline
        \texttt{\# . \# . \# . . . \#} \newline
        \texttt{\# \# \# \# \# \# \# \# \#} \newline

        La salida correcta sería: \newline
        \texttt{10}

    % 2. Explicar de forma clara, sencilla, estructurada y concisa, las ideas desarrolladas para la resolución del problema. Utilizar pseudocódigo y lenguaje coloquial (no código fuente). Justificar por qué el procedimiento resuelve efectivamente el problema.
    \subsection{Solución propuesta}
        Este problema resulta más simple de entender cuando se piensa el mapa como un grafo dirigido en tres dimensiones, donde la cantidad de nodos será de $FC(P+1)$ y en cada nivel habrá $FC$ nodos. De esta manera, podemos ver a cada piso del grafo (comenzando en el piso 0) como la cantidad de parades que se rompieron hasta ese punto. Los nodos estarán conectados cada uno con sus vecinos, aunque si uno de ellos es una pared, entonces el nodo se conectará con el vecino que representa a la pared pero un nivel más arriba. Hay que tener en cuenta que los nodos se modelan, para cada nivel, simplemente como el número de nodo en el grafo base (el del nivel 0) más la cantidad de nodos de cada nivel, y sus conexiónes (o aristas) se almacenan en listas de adyacencias.

        Teniendo el mapa representado de esta manera, sabemos en todo momento que la cantidad de paredes derribadas es menor o igual al límite establecido, por lo que si se puede llegar a la $x$ destruyendo una cantidad menor o igual de paredes que $P$, entonces existe un camino entre el origen y el destino.

        Para encontrar el camino mínimo entre $o$ y $x$ utilizamos el algoritmo conocido como \textit{Breadth-first Search} o \textit{BFS}, el cual conciste en recorrer el grafo analizando una sola vez cada nodo y desde éllos, observar sus vecinos. Si los vecinos de ese nodo aún no fueron analizados, se agregan a una cola para ser revisados después. Cada vez que se observa un vecino, chequeamos que no haya sido visto con anterioridad y si ese es el caso, se marca como visto cargando su distancia al origen en una \textit{lista de distancias} y agregándolo a la cola. Este algoritmo funciona en grafos donde las aristas no tienen peso o sus pesos son todos iguales, como en este ejercicio que como se dijo anteriormente, avanzar en el mapa implica sumar 1 a la cantidad de pasos, o bien aumentar en 1 la distancia al origen. Luego, los pesos de las aristas serán todos de valor 1 y puede utilizarse BFS.

        La \textit{lista de distancias} tiene el tamaño de la cantidad de nodos totales y cada posición $i$ representa al nodo $i$ del grafo. Esta lista solamente contiene la distancia al origen de cada nodo, inicializándose con $-1$ para todos y a medida que avanza el algoritmo, si el vecino $k$ del nodo $i$, que es el que se está analizando tiene distancia menor a $0$, entonces quiere decir que aún no ha sido observado.


        \subsubsection{Detalles implementativos}
            Para poder manejar correctamente el grafo, se creó la clase ListaAdy con namespace Grafos, la cual modela un grafo representado con listas de adyacencias. Las únicas operaciones que tiene la clase son el constructor, que recibe la cantidad de nodos totales, agregarArista, que recibe dos enteros $u,v$ y agrega $v$ a la lista de adyacencias de $u$, y BFS que recibe el número de nodo origen $s$, el número de origen de destino $t$, y la cantidad $f$ de filas y $c$ de columnas.
            La forma en que se almacenan las listas de adyacencias es con un vector de vectores de enteros, tipo que se provee de la librería estándar de C++.

            Normalmente, BFS solo necesitaría el nodo de origen y de destino, y almacenaría todos las posibles distancias de caminos que hay entre ambos nodos. Sin embargo, agregamos estos dos valores para el algoritmo termine la primera vez que encuentre el nodo destino. Esto es por cómo funciona BFS: Al recorrer todos los nodos según su distancia al origen, siempre se analizan los nodos que solo están a distancia 1 más que el nodo vecino "padre" (o sea, el nodo que tenía como vecino al que se está analizando y por el cual se agregó a la cola de revisión). Ergo, si existe un camino entre $s$ y $t$, BFS encontrará el más corto antes que cualquier otro y por eso es que podemos terminar su ejecución cuando eso pase. En peor caso, no existirá un camino entre $s$ y $t$, y BFS recorrerá todos los nodos.

            El algoritmo que resuelve el problema puede separarse en dos partes: Una se encarga de leer el mapa de entrada y construir el grafo dirigido tridimensional y la otra, es el BFS que busca verdaderamente la solución.
            En la primera parte recorremos el mapa que se encuentra guardado en una matriz de \textbf{char} $P+1$ veces, para así poder construir los $p+1$ niveles del grafo y poder conectar cada nodo con sus respectivos vecinos, y para poder verificar que el mapa pasado sea válido y buscar los nodos marcados con la $o$ y la $x$.
            La segunda parte se recorre el grafo una sola vez dado el funcionamiento de \textit{BFS}, pero como tiene $FC(P+1)$ nodos, entonces esa será la cantidad de nodos que pasarán por la cola de revisión como máximo. La función BFS está implementada de la siguiente manera:


            \begin{codesnippet}
            \begin{verbatim}
BFS(enteros : s, t, f, c)
  res = -1
  cola = cola vacía
  distancias[nodosTotales]
  para i entre 0 y nodosTotales-1
    distancias[i] = -1
  fin para

  cola.encolar(s)
  distancias[s] = 0
  mientras (cola no esté vacía)
    tope = cola.tope
    cola.desencolar //al ser una cola, desencola el tope de la misma
    si (tope % (f*c)) == t
      devolver distancias[tope]
    fin si

    para i entre 0 y largo(vecinos(tope))
      si distancias[vecinos(tope)[i]] < 0
        distancias[vecinos(tope)[i]] = distancias[tope] + 1
        cola.encolar(vecinos(tope)[i])
      fin si
    fin para
  fin mientras

  devolver res
            \end{verbatim}
            \end{codesnippet}

            El único cambio que tiene esta implementación de BFS respecto de la original, es que chequea si el resultado se encuentra antes de terminar, lo cual puede hacerse por lo antes dicho. Como complemento, podemos decir que en el código puede observarse que $distancias[tope]$ siempre existe y es mayor a $0$ porque antes de agregar un nodo a la cola, se carga en $distancias$ su valor correspondiente. Además, la razón por la cual se chequea el resto de dividir $tope$ por $f*c$ es que t es el número del nodo con la $x$ en el nivel 0, mientras que al ser un grafo tridimensional donde cada nivel representa la cantidad de paredes rotas desde el origen, $t$ va a estar en todos los niveles con una diferencia de $f*c*nivelActual$. Ergo, al dividir tope por $f*c$, el resto debería ser el número de nodo que se va a analizar como si fuera uno del nivel 0.



    % 3. Deducir una cota de complejidad temporal del algoritmo propuesto y justificar por qué el algoritmo cumple la cota dada. Utilizar el modelo uniforme.
    \subsection{Complejidad teórica}

      Para este análisis, nuevamente separaremos el algoritmo en dos partes (construcción del grafo y BFS). En la primera parte, se generan $P+1$ niveles de $FC$ nodos que como se explica en el punto anterior, se representan mediante listas de adyacencias. Agregar una arista entre dos nodos cuesta $O(1)$ ya que es simplemente agregar el número de nodo destino de la arista a la lista de adyacencia del nodo origen. Hacer esto por cada nodo costaría $O(|E|)$ por cada nivel con $|E|$ siendo el número de aristas. Sin embargo, al ser un digrafo tipo \textit{grid}, la cantidad máxima de aristas por nodo es de 4 porque cada nodo se conecta con, a lo sumo, un nodo a cada lado (izquierda, derecha, arriba y abajo). Además, si bien hay $FC$ nodos en cada nivel, los nodos que sean paredes no tendrán aristas de entrada que provengan de nodos del mismo piso. Luego, $|E| \leq 4FC \in O(FC)$. Como esto ocurre $P+1$ veces, la complejidad de la primera parte es

      \[
        O((P+1)FC) \in O(FCP)
      \]


      La segunda parte, el algoritmo conocido como \textit{BFS}, tiene complejidad $O(|V| + |E|)$ siendo $|V|$ la cantidad de nodos. Ya probamos que $|E| \in O(FCP)$ y sabemos que $|V| = FC$, entonces la complejidad de la segunda parte es

      \[
        O(FC + FCP) \in O(FCP)
      \]

      Considerando las dos partes juntas, nos queda que la complejidad temporal es

      \[
        O(FCP + FCP) \in O(FCP)
      \]

      En cuanto a la complejidad espacial, el tamaño del grafo sobre listas de adyacencias es de $O(FCP)$ ya que es la cantidad de nodos totales sumado a lo que, por lo probado anteriormente, acota la cantidad de aristas totales.


    % 4. Dar un código fuente claro que implemente la solución propuesta. Se deben incluir las partes relevantes del código como apéndice del informe impreso entregado.

    % 5. Realizar una experimentación computacional para medir la performance del programa implementado. Usar un conjunto de casos de test en función de los parámetros de entrada, con instancias aleatorias e instancias particulares (de peor/mejor caso en tiempo de ejecución, por ejemplo). Presentar en forma gráfica una comparación entre los tiempos medidos y la complejidad teórica calculada y extraer conclusiones.
    \subsection{Experimentación}

	Para poder mostrar que la cota propuesta en la complejidad temporal funciona para el algoritmo que resuelve este problema, realizamos experimentos con cantidades de personas entre 1 y 7 pero siempre con mayor o igual número de arqueólogos que de caníbales. Los casos de prueba pueden observarse en la tabla que se encuentra en el anexo de este informe.

  Los resultados obtenidos fueron plasmados en el siguiente gráfico. El mismo es la representación del tiempo en funcion de la cantidad de arqueólogos. También se muestra la funcion propuesta como cota de complejidad temporal.

  \begin{figure}[H]
      \begin{center}
        \includegraphics[width=0.7\columnwidth]{imagenes/exp1Ej1-1a7.jpeg}
        \caption{}
      \end{center}
  \end{figure}

  Para cada valor en $x$ puede observarse que a medida que crece, hay más puntos en $y$ para el mismo $x$. Esto es porque si bien el eje X es la cantidad de arqueólogos, también varía la cantidad de caníbales por lo que el tiempo que toma cada ejecución del programa también depende de este valor, pero como se aclaró en la sección de complejidad, el número de caníbales va de 0 a la cantidad de arqueólogos (porque en caso contrario el programa termina en seguida a menos que no haya arqueólogos, caso que se verá más adelante) lo cual implica que nuestro tamaño de entrada $n$ (la cantidad de arqueólogos) es en realidad a lo sumo $2n$, pero en términos de complejidad el $2$ es una constante que podemos sacar.
  De acuerdo a la tabla proporcionada, los pares (arqueólogo, caníbal) que toman más tiempo son \texttt{(1,1), (2,2), (3,2), (4,2), (5,3), (6,3), (7,3)}. Sin embargo, podemos notar como la cota de complejidad cumple, aunque no de manera ajustada, su función para estos experimentos.


  Para el caso en que no haya arqueólogos y solo haya caníbales, esperábamos que la resolución del problema sea más rápida que en los casos que hay más arqueólogos que caníbales. Probamos con cantidades de caníbales entre 1 y 200 y en el próximo grafico se ilustran los resultados de tiempo en función del número de personas.

  \begin{figure}[H]
      \begin{center}
        \includegraphics[width=0.7\columnwidth]{imagenes/exp2Ej1.jpeg}
        \caption{}
      \end{center}
  \end{figure}

  Notamos como el tiempo que toma a mayor cantidad de caníbales sin arqueólogos crece mucho más lento que para los casos con arqueólogos. Esto se debe a que por un lado, las ramas en que cruzan arqueologos no se prueban, sino que solo se intenta que crucen 1 o 2 caníbales. Luego, la cantidad de estados posibles se reduce a 2 veces la cantidad de formas que se pueden distribuir los caníbales en ambos lados del puente (una por cada lado de la linterna), que es igual a $2*(n^2)$; reducimos la cota de complejidad a $O(n $log$ n*2^{n^2})$. El cambio no es muy grande debido a que las cotas no están totalmente ajustadas, pero funcionan para dar una idea del peor caso acotado por arriba.
