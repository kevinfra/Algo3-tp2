\section*{Informe de modificaciones}

A continuación se señalan los cambios realizados para la reentrega del trabajo:

\begin{itemize}
	\item{
		Ejercicio 1
		\begin{itemize}
			\item Se agrega pseudocódigo a la sección de detalles implementativos, para mejorar
			la comprensión de la implementación y la demostración de su complejidad.
			\item Para la experimentación se generaron de forma lineal en lugar
			de exponencial los tiempos de ejecución.
			\item En los gráficos de la experimentación se ignoraron los primeros
			valores para no tener picos al comienzo. Esto además se aclara en la
			descripción de la sección.
		\end{itemize}
	}
	\item{
		Ejercicio 2
		\begin{itemize}
			\item En los gráficos de la experimentación se ignoraron los primeros
			valores para no tener picos al comienzo. Esto además se aclara en la
			descripción de la sección.
		\end{itemize}
	}
	\item{
		Ejercicio 3
		\begin{itemize}
			\item En la experimentación se reemplazó la cota de complejidad de $N^{N+2}$ por $N^3 \times N!$. Esto incluye la función utilizada como cota en los gráficos.
			\item En la experimentación se reemplazó la constante de la cota teórica $0.001$ por $0.1$.
			\item Se agregó pseudocódigo en la explicación de las podas para una mejor comprensión.
		\end{itemize}
	}
\end{itemize}
