\section{Ejercicio 3: Si esto no es el tesoro, el tesoro dónde está}
    % 1. Describir detalladamente el problema a resolver dando ejemplos del mismo y sus soluciones.
    \subsection{Descripción del problema}

		\begin{figure}[ht]
			\begin{center}
				\includegraphics[width=0.4\columnwidth]{imagenes/tesoros.jpg}
				\caption{Fortuna y gloria muñeca, fortuna y gloria}
			\end{center}
		\end{figure}

        Una vez equilibrada la balanza, las paredes vuelven a su lugar. Con la llave obtenida logran abrir la puerta que estaba trabada. \par
		Al iluminar la habitación se encuentran con una sala de N tesoros. \par
		Por cada tipo de tesoro i, hay $C_{i}$ cantidades, con un valor $V_{i}$ y un peso $P_{i}$. Si bien el objetivo principal de la expedición era otro, nunca viene mal armarse de algún recuerdo. 
		Tienen M mochilas disponibles para llevarse los tesoros y cada una tiene una capacidad máxima de peso $K_{j}$ que puede llevar. \par
		El objetivo es llevarse el mayor valor posible de tesoros. Para esto, se debe devolver como resultado el valor total de tesoros guardados y además, por cada mochila, la cantidad de tesoros llevados y de qué tipo son.

        Por ejemplo, para la siguiente entrada:
        
        \begin{verbatim}
        M = 2
        K = { 1, 3 }
        
        N = 3 
        C = { 1, 2, 3 }
        V = { 2, 4, 10 }
        P = { 1, 2, 5 }
        \end{verbatim}

        Una posible salida válida sería:

        \begin{verbatim}
        S = 6
        M1 = { 1, 1 } 
        M2 = { 1, 2 }
        \end{verbatim}

        Es decir, para el caso en el que tenemos dos mochilas con capacidades 1 y 3, y nos encontramos con tres tesoros, el primero de peso 1 y valor 2, el segundo de peso 2 y valor 4, y el tercero de peso 5 y valor 10... \par Nos llevaremos en la primer mochila el primer tesoro y en la segunda el segundo tesoro. El tercer tesoro ya no nos entra.

    % 2. Explicar de forma clara, sencilla, estructurada y concisa, las ideas desarrolladas para la resolución del problema. Utilizar pseudocódigo y lenguaje coloquial (no código fuente). Justificar por qué el procedimiento resuelve efectivamente el problema.
    \subsection{Solución propuesta}
    Para solucionar el problema planteado se usó la técnica algorítmica de \emph{Programación dinámica}. \par

    La idea es separar el problema principal en subproblemas. En este caso, el subproblema es, en base a la capacidad restante de cada mochila y, al valor y peso de cada tesoro, en qué mochila conviene meter cada uno (incluyendo la opción de no meterlo en ninguna). Más formalmente el subproblema es: si ya resolvimos el beneficio máximo para los tesoros $T_{j}$, $j < i$, cómo se calcula el beneficio máximo teniendo en cuenta al $T_{i}$. Entonces la idea es la siguiente: si tenemos un solo tesoro (caso base) el máximo beneficio que se puede obtener es el valor del tesoro en caso de que entre en alguna mochila. Para el tesoro siguiente (recursión) el máximo beneficio que se puede obtener es el máximo entre: el valor del tesoro actual más el valor máximo que se podía llevar con el tesoro anterior con el peso restante de la mochila 1 al colocar este tesoro si entra, a esto lo llamaremos \textbf{ValorMochila1}, el mismo valor pero para la mochila 2 o \textbf{ValorMochila2}, el mismo valor pero para la mochila 3 o \textbf{ValorMochila3} , y así sucesivamente para las n mochilas y el valor que se obtiene si no se coloca, que sería el valor que se puede obtener para el tesoro anterior. Es decir la idea intuitiva es que en cada paso se compara el benficio de poner el tesoro en algunas de las mochilas con el beneficio de no ponerlo, en base al beneficio anterior. Podría verse como que se van comparando de a pares el beneficio de un tesoro y otro y nos quedamos con el mejor, siempre viendo todas las posilibidades de ponerlo en todas las mochilas, y despues de quedarse con la mejor opcion, se repite el procedimiento pero para el beneficio máximo calculado entre esos dos tesoros y uno que sigue. Se puede observar que el orden en el cual se recorren los tesoros no es importante. La solución que se obtiene luego de comparar con el último tesoro es óptima ya que en cada paso se está obteniendo el mayor beneficio,comparando todas las posibilidades. Se propone una función de recursión, la cual está definida en el rango 1..n, siendo n la cantidad de tesoros total y $\digamma(i)$ indicando el beneficio máximo que se puede obtener considerando los primeros i tesoros: \par

    $\digamma(1)$ = valor $v$ del primer tesoro si éste entra en alguna de las mochilas

    $\digamma(i)$ = max(valorMochila1,valorMochila2,valorMochila3, $\digamma(i-1)$)

    Lo que falta aclarar en esta función es que para cada tesoro se debe calcular el beneficio máximo que se puede obtener para cada peso posible de cada mochila, es decir se calculan los beneficios máximos para todos los posibles pesos de una mochila, desde uno hasta el peso P de la mochila, para que sea posible conocer cuál es el beneficio total que se puede obtener con el próximo tesoro, ya que se necesitaba para eso el beneficio que se puede obtener con el peso restante al poner un tesoro. 

    Por lo tanto, el algoritmo propuesto consta de una matriz para cada tesoro, de tamaño $\[\prod_{i=0}^{3}C_{i} + 1$, siendo $C_{i}$ las capacidades de las \textbf{M} mochilas, donde $M \leq 3$. Y además una matriz nula de igual dimensión que servirá para el caso en donde no hayan tesoros. \par

    El algoritmo funciona de la siguiente manera:
    1) Se crea una matriz de $\[\prod_{i=0}^{M}C_{i} + 1$ para cada tesoro, y una extra inicial con los valores de todas las posiciones inicializados en 0.

    \begin{codesnippet}
    \begin{verbatim}
    for i = 1 .. cantTesoros + 1
        for j = 1 .. capacidadMochila1 + 1
            for k = 1 .. capacidadMochila2 + 1  
                for l = 1 .. capacidadMochila3 + 1  
                    matriz[i][j][k][l] <-- 0
                end
            end
        end
    end
    \end{verbatim}
    \end{codesnippet}


    2) En el caso donde $M < 3$, el programa se encarga de fijar las capacidades de las $3 - M$ mochilas en 0.
    Para cada posición (i, j, k) de la matriz del tesoro t, siendo i, j y k la capacidad \emph{restante} de las mochilas 1, 2 y 3, se pregunta si t entra en alguna de las tres mochilas.
    Si entra en la mochila 1, se obtiene valor(t) sumado al valor de la matriz del tesoro t-1 (recordar que el primer tesoro tiene una matriz predecesora con todos sus valores nulos) en la posición correspondiente según la capacidad restante de esa mochila (i - peso(t), j, k).
    Se repite este procedimiento para cada mochila. 
    Y además se obtiene el valor de la posición (i, j, k) de la matriz del tesoro t-1 para incluir el caso en el que no se mete el objeto en la mochila.
    Luego se obtiene el resultado mayor de cada suma, y se guarda en la matriz del tesoro t, en la posición (i, j, k)..

    \begin{codesnippet}
    \begin{verbatim}

    valorMochila1 = 0
    valorMochila2 = 0
    valorMochila3 = 0

    valorNingunaMochila = matriz[t-1][i][j][k]


    if (pesoTesoro <= i ||pesoTesoro <= j || pesoTesoro <= k)

        if pesoTesoro <= i then
            valorMochila1 <-- valor(t) + matriz[t-1][i - pesoTesoro][j][k]
        endif

        if pesoTesoro <= j then
            valorMochila2 <-- valor(t) + matriz[t-1][i][j - pesoTesoro][k]
        endif

        if pesoTesoro <= k then
            valorMochila3 <-- valor(t) + matriz[t-1][i][j][k - pesoTesoro]
        endif
    
        matriz[t][i][j][k] <-- max(valorMochila1, valorMochila2, valorMochila3, valor(t-1))
    else
        matriz[t][i][j][k] < -- valor(t-1)


    

    \end{verbatim}
    \end{codesnippet}

    Se repite este procedimiento para cada tesoro, hasta completar todas las posiciones de cada matriz.
    El último valor obtenido de la matriz del último tesoro, será el valor máximo que se puede acumular en las mochilas con los tesoros.

    3) Luego, empezando desde la matriz del último tesoro, desde la última posición (i, j, k), se pregunta si entra ese tesoro en alguna mochila. 
    Si no entra, se repite el procedimiento para el tesoro anterior.
    Si entra, se pregunta en cuáles de las mochilas entra ese tesoro. Para eso se guarda la capacidad disponible de cada mochila (inicialmente con su capacidad total) y se pregunta si el tesoro tiene peso menor o igual a cada capacidad.
    En caso de que entre en más de una mochila, se compara el valor resultante de ponerlo en cada una. Por ejemplo, para la mochila 1, se obtiene el valor de la posición (i - peso(t), j, k) de la matriz del tesoro t-1.
    Una vez obtenido en qué mochila es conveniente poner a ese tesoro, se almacena en cada vector de salida (hay uno por mochila) los tipos de tesoro que guardo en cada mochila. Se incrementa el contador de tesoros de esa mochila, y luego el indice pasa a ser aquella posición donde estaba el valor máximo. Por ejemplo, si la mejor opción es guardar ese tesoro en la mochila 1, (i, j, k) pasa a ser (i - peso(t), j, k), t pasa a ser t-1, y se le resta peso(t) a la capacidad restante de esa mochila.

    \begin{codesnippet}
    \begin{verbatim}

    entraEnAlgunaMochila <-- matriz[t+1][i][j][k] != matriz[t][i][j][k]

    if entraEnAlgunaMochila then
    	valorAnterior1 <-- 0
    	valorAnterior2 <-- 0
        valorAnterior3 <-- 0
	    entraEnMochila1 <-- (pesoTesoro <= i);
		entraEnMochila2 <-- (pesoTesoro <= j);
        entraEnMochila3 <-- (pesoTesoro <= k);

		if entraEnMochila1 then
			valorAnterior1 <-- matriz[t][i - pesoTesoro][j][k]
		endif
		if entraEnMochila2 then
			valorAnterior2 <-- matriz[t][i][j - pesoTesoro][k]
		endif
        if entraEnMochila2 then
            valorAnterior3 <-- matriz[t][i][j][k - pesoTesoro]

		if entraEnMochila1 && valorAnterior1 >= valorAnterior2  && valorAnterior1 >= valorAnterior3
        then
			add(mochila1, tipo(t))
			cantTesorosMochila1++
			i -= peso(t)
		else if entraEnMochila2 && valorAnterior2 >= valorAnterior3 
			add(mochila2, tipo(t))
			cantTesorosMochila2++
			j -= peso(t)
        else
            add(mochila3, tipo(t))
            cantTesorosMochila3++
            k -= peso(t)
		endif

	endif

    \end{verbatim}
    \end{codesnippet}

    Se repite este procedimiento para cada tesoro.

    4) Una vez finalizados los primeros tres pasos, se copia en una tupla el valor obtenido entre todos los tesoros guardados en las mochilas, y un vector mochilas que contiene un vector por mochila. Cada uno de estos vectores guarda la cantidad de tesoros y los tipos de tesoro que hay en esa mochila.

    % Para incluir una tercera mochila, se extendió la matriz a 3 dimensiones (4 para ser correctos, ya que una primera dimensión es para agrupar en un mismo vector la matriz de cada tesoro). Y cada vez que se realiza una comparación de valores máximos, se compara también con los valores de la tercer mochila.



 %    JULI:
 %    El objetivo de hacer esto, es tener en cuenta todos los casos y valores acumulados, de poner o no cada tesoro, y en qué mochila se pone. Utilizando en todo caso el mayor valor de tesoros acumulados. En los campos de la matriz ira el valor que queda de poner el tesoro o no, y las capacidades restantes de las mochilas. \par
 %    El algoritmo se divide en dos partes, una parte es completar las matrices de tesoros, con el maximo valor posible que se puede acumular en cada mochila según el peso disponible. La segunda parte, consiste en definir, qué tesoros van en cada mochila, ubicando según el valor restante, de poner el tesoro en una mochila u otra.

 %    Para esto, uno puede pensarlo como una matriz de tres dimensiones o un cubo por tesoro, además de un cubo nulo. Por cada tesoro que se analiza, los pasos son iguales. Si la capacidad de las mochilas, representado por el índice (i,j,k) que las recorre (i representa la capacidad de la M1, j de la M2 y K de la M3) es menor que el peso del tesoro, el valor obtenido es igual al acumulado en el tesoro anterior (salvo el primer caso, en donde el tesoro anterior es la matriz nula). Si entra en alguna mochila, uno debe fijarse, que valor restante se obtiene, de restarle a su índice, el peso del objeto. Y, en la posición de la matriz se deja el máximo de ponerlo en alguna mochila o no ponerlo.\par

 %    Una vez explicado brevemente, pasaremos a detallar el algoritmo, dividiéndolo en dos secciones. La primera, completa las matrices de los tesoros, según el valor máximo posible por capacidad. Mientras que la segunda sección, recorre de atras para adelante las matrices, para ubicar los tesoros.\par

 %    El siguiente pseudocodigo mostrara como se completan las matrices,

 %    \begin{codesnippet}
	% \begin{verbatim}
 %    cuboMagico = Inicializo un vector de matrices de Int
 %    Inicializo cantTesoros+1 matrices de (N + 1)*(M + 1)*(K +1) y las guardo en el vector cuboMagico.
 %    //N, M y K capacidades de las mochilas.
 %    for(iTesoroActual= 0...cuboMagico.size()) do
 %        //Recorro todas las matrices de tesoros
 %        iTesoro = El indice a la matriz del tesoro anterior.
 %        valorTesoro = tesoroValor[iTesoroActual]
 %        pesoTesoro = tesoroPeso[iTesoroActual]
 %        valorNingunaMochila = 0; //Me guardo el valor por si no entra en ninguna mochila
 %        //Recorro todas las matrices de tesoros
 %        for(iMochila1 = 0 .. N + 1) do
 %            for(iMochila2 = 0 .. M + 1) do
 %                for(iMochila3 = 0 .. N + 1) do
 %                    if(pesoTesoro entra en alguna mochila) do
 %                        if(pesoTesoro entra en mochila M_{1})
 %                            valorMochila1 = valorTesoro + cuboMagico[iTesoro][iMochila1-pesoTesoro][iMochila2][iMochila3]; 
 %                            //Valor de la matriz del tesoro anterior.
 %                        endif
 %                        if(pesoTesoro entra en mochila M_{2}) do
 %                    	    valorMochila2 = valorTesoro +  cuboMagico[iTesoro][iMochila1][iMochila2 - pesoTesoro][iMochila3];
 %                    	    //Valor de la matriz del tesoro anterior.
 %                        endif
 %                        if(pesoTesoro entra en mochila M_{3}) do
 %                            valorMochila3 = valorTesoro + cuboMagico[iTesoro][iMochila1][iMochila2][iMochila3 - pesoTesoro];
 %                            //Valor de la matriz del tesoro anterior.
 %                        endif
 %                    cuboMagico[iTesoro][iMochila1][iMochila2][iMochila3] = max(valorNingunaMochila, valorMochila1, valorMochila2, valorMochila3);
 %                    //Me quedo con el maximo de ponerlo en alguna mochila o no ponerlo.
 %                    iTesoro ++ //Avanzo el indice al tesoro anterior.
 %                end
 %            end
 %        end

	% \end{verbatim}
	% \end{codesnippet}






	% Para la segunda parte, se decide por cada tesoro, en qué mochila irá. Para esto, se recorre desde la última posición de la matriz del último tesoro. En esta posición, se encuentra el mejor valor acumulado posible. La decisión se toma de la siguiente manera
	% 	%Listita:
	% 	 A) En qué mochilas entra?
	% 	 B) Para cada mochila en la cual entra. Qué valor restante se encuentra en la posición de la matriz del del tesoro anterior en los pesos restantes, de ponerlo en cada una de las mochilas.
	% 	 C) En el caso que el valor restante sea igual, lo pongo en cualquiera de las dos
	% 	 D) En el caso que haya un máximo, ubico el tesoro en la mochila que obtiene el valor máximo restante.
	% Sigo este algoritmo por cada matriz, hasta no tener capacidad disponible en ninguna de las mochilas.

	% Si se analiza la forma en que se construye la solución, tanto en la primera como en la segunda etapa, se tienen en cuenta cuál es el valor máximo posible acumulado, de ubicar los tesoros. Es por esto que el resultado, siempre dará la suma máxima posible.
	% FIN JULI

   
    \subsection{Complejidad teórica}
        Por lo visto anteriormente, el algoritmo tiene complejidad temporal de $O(\prod_{i=0}^{C} (K+1)^M)$ donde $M$ es la cantidad de mochilas, $K$ es la capacidad de la mochila más grande y $C$ es la cantidad de tesoros.
        En cuanto a la complejidad espacial, resulta también ser la misma.
    

    % 4. Dar un código fuente claro que implemente la solución propuesta. Se deben incluir las partes relevantes del código como apéndice del informe impreso entregado.

    % 5. Realizar una experimentación computacional para medir la performance del programa implementado. Usar un conjunto de casos de test en función de los parámetros de entrada, con instancias aleatorias e instancias particulares (de peor/mejor caso en tiempo de ejecución, por ejemplo). Presentar en forma gráfica una comparación entre los tiempos medidos y la complejidad teórica calculada y extraer conclusiones.
    \subsection{Experimentación}

        Al igual que con los otros dos ejercicios, se realizaron pruebas experimentales para verificar que el tiempo de ejecución del algoritmo cumpliera con la cota asintótica de $O((\prod_{i=0}^{M} K_i) * C + C)$, teóricamente demostrada para el peor caso. Las pruebas que se realizaron consistieron de probar con distinta cantidad de tamaños (1 a 30) para 1 a 3 mochilas, donde la cantidad de tesoros era igual a los tamaños de las mochilas. El objetivo era observar el tiempo que toma el algoritmo cambiando estos valores. Para cada tesoro, su peso y valor fue un número random entre 1 y el tamaño de la mochila, tomado utilizando random de la librería estandar y con una distribución uniforme.
        El resultado de este experimento puede visualizarse en este gráfico:


      \begin{figure}[H]
      \begin{center}
        \includegraphics[width=0.7\columnwidth]{imagenes/exp1Ej3.jpeg}
        \caption{}
      \end{center}
      \end{figure}

        Notamos que la línea que corresponde al caso de una mochila sola no puede verse ya que al ser una productoria de los tamaños de las mochilas, ésta solo llega como máximo a 30.