\section{Introducción}

En este trabajo se tienen tres problemas que se resolvieron aplicando las
técnicas algorítmicas estudiadas en la materia. Los mismos presentaron cada uno un desafío
distinto, teniendo que aplicar métodos diferentes para resolverlos y cumplir
con los requisitos exigidos.

Además de la resolución de los mismos, se procedió a demostrar la correctitud de
cada implementación. Esto fue acompañado a su vez de una justificación de la
complejidad temporal.

Cada ejercicio contó con su respectiva experimentación para corroborar que la
complejidad temporal teórica se cumpliera y en los casos donde el algoritmo
podía comportarse mejor, verlo reflejado de alguna manera.

Los experimentos contaron con diversas medidas para asegurar su efectividad de
las cuales las siguientes fueron iguales para los tres problemas:
\begin{itemize}
	\item{Sólo se midió el costo temporal de generar la solución, no
			de lectura y escritura del problema.}
	\item{Para la medición del tiempo se utilizó la biblioteca \texttt{chronos}
			con unidad de tiempo en microsegundos.}
\end{itemize}
